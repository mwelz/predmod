\documentclass[11pt]{article}
\usepackage[utf8]{inputenc}
\usepackage{amsmath, bm, amssymb, hyperref, xcolor, caption, dsfont}

% link coloring
\hypersetup{
	hidelinks,
    colorlinks = true,
    linkbordercolor = {white},
    citecolor = blue,
    urlcolor = blue,
    linkcolor = blue,
    linktoc = all 
}

% layout
\usepackage[
  top=2cm,
  bottom=2cm,
  left=1.5in,
  right=1.5in,
  headheight=17pt, % as per the warning by fancyhdr
  includehead,includefoot,
  heightrounded, % to avoid spurious underfull messages
]{geometry} 

% Figure caption options
\captionsetup{labelfont = {bf, sc, color=blue}} % figure label 
\captionsetup{font = {singlespacing, small}} % caption

%%  bib settings
\usepackage{natbib}
\setcitestyle{authoryear} 
\bibliographystyle{apalike} 

% author information
\title{\textsc{Survival Models}}
\author{Max Welz \\
  \href{mailto:welz@ese.eur.nl}{\texttt{welz@ese.eur.nl}}}
  
\date{%
    \textsc{Econometric Institute\\ Erasmus School of Economics}\\[2ex]%
    \today}
    
% definitions
\newcommand{\E}{\mathbb{E}}
\renewcommand{\P}{\mathbb{P}}
\newcommand{\R}{\mathbb{R}}
\newcommand{\N}{\mathbb{N}}
\newcommand{\V}{\mathbb{V}}
\newcommand{\D}{\mathcal{D}}
\newcommand{\e}{\text{e}}
\renewcommand{\d}{\text{d}}
\newcommand{\indic}{\mathds{1}}


\begin{document}

\maketitle

\section{Essential Theory}
Suppose we want to model the failure rate of a some statistical process. Concretely, let the random variable $Y$ denote the (non-negative) real-valued failure time of the process of interest. Let $F$ be the distribution function of $Y$ and $f$ be its corresponding density. By definition, for some time $t\in[0, \infty]$,
\[
    F(t) = \P[Y \leq t]
    =
    \int_0^t f(s)\d s
\]
measures the probability that the process fails before or at time $t$. Conversely, the survival function $S$, defined by
\[
    S(t) = 1 - F(t) = \P[Y > t],
\]
is the probability that failure occurs \textit{after} time $t$. A popular choice for $F$ is the exponential distribution. An essential quantity in survival modeling is the \textit{hazard function} $h:\R\to\R$, defined by
\[
    h(t) = \frac{f(t)}{1-F(t)} = \frac{f(t)}{S(t)}.
\]
We emphasize that the hazard function is \textit{not} a probability, which is a frequent misconception. For some time $t\in[0, \infty]$, the \textit{cumulative hazard function} of hazard $h$ is defined by
\[
    H(t) = \int_0^t h(s) \d s.
\]

Observe that the definition of the hazard function $h$ gives rise to a differential equation of distribution $F$, namely $h(t) = \frac{F'(t)}{1-F(t)}$. Solving yields the following useful identity, which expresses distribution $F$ in terms of cumulative hazard $H$:
\[
    F(t) = 1 - \exp\big( - H(t) \big). 
\]
Thus, we can also express survival $S$ in terms of  cumulative hazard $H$:
\[
    S(t) = \exp \big( - H(t) \big).
\]

\section{Cox Proportional Hazard Modeling}
A \textit{Cox proportional hazard model} (\cite{cox1972}; hereafter Cox model) attempts to explain the survival time $Y$ by some explanatory variables which are collected in a $p$-dimensional random vector $X$. An essential component of a Cox model is the baseline hazard function $h_0$, which is a pre-specified hazard function that only depends on time instead of variables in $X$. The corresponding cumulative baseline hazard and baseline survival functions of $h_0$ are denoted by $H_0$ and $S_0$, respectively. Often, $h_0$ is chosen to be modeled non-parameterically, for instance via a Kaplan-Meier or a Nelson-Aalen estimator (see \cite{cameron2005} for definitions of these estimators).

With pre-specified baseline hazard and time $t \in[0,\infty]$, the hazard function of a Cox model is given by the semi-parameteric expression
\begin{equation}\label{eq:cox-hazard}
    h_\beta(t | X) = h_0(t) \exp \big( X^\top \beta \big),
\end{equation}
where $\beta = (\beta_1,\dots,\beta_p)\in\R^p$ is a fixed but unknown vector of coefficients. Observe that we make the hazard's dependence on the vectors $X$ and $\beta$ explicit by expressing it as a conditional function of $X$ and by using $\beta$ as a subscript, respectively. By definition, it holds for the cumulative hazard $H_\beta$ of $h_\beta$ that
\begin{equation*}
\begin{split}
    H_\beta(t|X) &= \int^t_0 h_0(s) \exp \big(X^\top \beta \big) \d s
    \\
    &= \exp \big(X^\top \beta \big) H_0(t)
\end{split}
\end{equation*}
Thus, the associated survival function $S_\beta$ of a Cox model is by definition given by
\begin{equation} \label{eq:cox-survival}
\begin{split}
    S_\beta(t) 
    &=
    \exp\Big( - \exp\big( X^\top \beta \big) H_0(t)  \Big)
    \\&=
    \Big(\exp\big( - H_0(t) \big) \Big)^{\exp(X^\top \beta)}
    \\&=
    S_0(t)^{\exp(X^\top \beta)}.
\end{split}
\end{equation}

\section{Fitting a Cox Proportional Hazard Model}
\subsection{Setup}
Suppose we have information on $n$ observations, indexed $i=1,\dots,n$. Suppose further that we observe non-negative real-valued random variables $Y_i$ that measure the failure time of observation $i$, as well as $p$-dimensional random vectors $X_i$ which contain explanatory variables for the $Y_i$. Assume that the failure times $Y_i$ are right-censored at a fixed time $T>0$, meaning that observations which have not yet failed at censoring time $T$ are counted as \textit{survivors}. Thus, for all survivors, it holds that $Y_i = T$, and for all failures, we have that $Y_i < T$. Thereupon, define a binary random variable $\delta_i$ that takes the value one if individual $i$ is a failure and the value zero if it is a survivor. The goal is to use the random sample $\big\{ (X_i, Y_i, \delta_i) \big\}_{i=1}^n$ to estimate the unknown coefficient vector $\beta\in\R^p$ in the Cox hazard function in \eqref{eq:cox-hazard}. We do so via maximum likelihood estimation. 

Assume for the moment that all failure times $Y_i$ are unique; see Section \ref{sec:non-unique-times} for a discussion on non-unique failure times. To perform maximum likelihood estimation, we consider the partial likelihood function $L$, which is calculated on the failures and constructed as
\begin{equation} \label{eq:cox-likelihood}
    \begin{split}
        L(\beta)
        &=
        \prod_{\{ i \in [n] : \delta_i = 1 \}}
        \frac{ h_\beta(Y_i | X_i) }{ \sum_{\{ j\in[n] : Y_j \geq Y_i \}} h_\beta(Y_i | X_j) }
        \\
        &=
        \prod_{\{ i \in [n] : \delta_i = 1 \}}
        \frac{ \exp(X_i^\top \beta) }{ \sum_{\{ j\in[n] : Y_j \geq Y_i \}} \exp(X_j^\top \beta) },
    \end{split}
\end{equation}
with the hazard function $h_\beta$ as in \eqref{eq:cox-hazard}. Observe that the the partial likelihood does \textit{not} depend on the baseline hazard $h_0$, as corresponding expressions cancel in the first line of the previous display. 

We maximize \eqref{eq:cox-likelihood} by maximizing its corresponding log-likelihood, minus a regularization penalty $P_\alpha$, which depends on some pre-specified $\alpha \in [0,1]$. The strength of the sparsity-enforcing penalty $P_\alpha$ is controlled via a fixed tuning parameter $\lambda_n \geq 0$. Thus, we obtain estimator $\widehat{\beta}$ of $\beta$ in \eqref{eq:cox-hazard} by solving
\begin{equation} \label{eq:cox-loss}
    \widehat{\beta} 
    =
    \arg\max_{\beta \in \R^p}
    \left\{
        \frac{2}{n}
        \left[
            \sum_{\{i\in[n]:\delta_i = 1\}}
            \left(
            X_i^\top \beta
            -
            \ln
            \left(
            \sum_{\{ j\in[n] : Y_j \geq Y_i \}}
            \exp
            \big(
            X_j^\top \beta
            \big)
            \right)
            \right)
        \right]
        -
        \lambda_n
        P_\alpha (\beta)
    \right\},
\end{equation}
where the scaling factor $2/n$ has been added for mathematical convenience, and $P_\alpha$ is the elastic net penalty \citep{zou2005}, defined by
\[
    P_\alpha (\beta) = \alpha \sum_{j=1}^p |\beta_j| + \frac{1}{2} (1 - \alpha) \sum_{j=1}^p \beta_j^2
    =
    \alpha \|\beta \|_1 + \frac{1}{2} (1-\alpha) \|\beta \|_2^2.
\]

The problem in \eqref{eq:cox-loss} is convex for all choices of $\alpha \in [0,1]$ and $\lambda_n \geq 0$, hence it can be solved easily. The value of tuning parameter $\lambda_n$ can be determined via cross-validation. Numerical details are described in \cite{simon2011}.

With estimate $\widehat{\beta}$, we can estimate the Cox model's survival function in \eqref{eq:cox-survival} by
\[
    \widehat{S}(t) = S_{\widehat{\beta}}(t)
    =
    \widehat{S}_0(t)^{\exp\big(X^\top \widehat{\beta}\big)}.
\]
Recall that the baseline survival $S_0$ does not depend on $\beta$, hence its estimate $\widehat{S}_0$ also does not depend on $\widehat{\beta}$. Thus, as previously discussed, $\widehat{S}_0$ is typically estimated separately in non-parameteric fashion. An estimate of the Cox model's hazard function in \eqref{eq:cox-hazard} can be constructed analogously.

\textcolor{red}{TODO: Add some stuff on the proportional hazard assumption and merge with main document}

\subsection{What if the Failure Times are not Unique?} \label{sec:non-unique-times}
Consider a situation where some of the failure times $Y_i$ are not unique. \cite{breslow1975} and \cite{efron1977} propose two different approaches for this situation. In the following, we briefly discuss the approach of \cite{breslow1975}.

Let the sets $\mathcal{D}_i = \{ j \in [n] : Y_j = Y_i \}$ contain the observations whose failure times are tied with the one of individual $i$. Then, the likelihood function $L$ in \eqref{eq:cox-likelihood} becomes
\begin{equation*} 
        L(\beta)
        =
        \prod_{\{ i \in [n] : \delta_i = 1 \}}
        \frac{ \sum_{\{j\in \mathcal{D}_i\}}\exp(X_j^\top \beta) }{ \Big( \sum_{\{ j\in[n] : Y_j \geq Y_i \}} \exp(X_j^\top \beta) \Big)^{|\mathcal{D}_i|} }
\end{equation*}
and the optimization problem in \eqref{eq:cox-loss} is adapted correspondingly.

\bibliography{bibliography}

\end{document}
