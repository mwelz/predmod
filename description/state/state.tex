\documentclass[11pt]{article}
\usepackage[utf8]{inputenc}
\usepackage{amsmath, bm, amssymb, hyperref, xcolor, caption, dsfont}

% link coloring
\hypersetup{
	hidelinks,
    colorlinks = true,
    linkbordercolor = {white},
    citecolor = blue,
    urlcolor = blue,
    linkcolor = blue,
    linktoc = all 
}

% layout
\usepackage[
  top=2cm,
  bottom=2cm,
  left=1.5in,
  right=1.5in,
  headheight=17pt, % as per the warning by fancyhdr
  includehead,includefoot,
  heightrounded, % to avoid spurious underfull messages
]{geometry} 

% Figure caption options
\captionsetup{labelfont = {bf, sc, color=blue}} % figure label 
\captionsetup{font = {singlespacing, small}} % caption

%%  bib settings
\usepackage{natbib}
\setcitestyle{authoryear} 
\bibliographystyle{apalike} 

% author information
\title{\textsc{State of the CITRUS Project}}
\author{Max Welz \\
  \href{mailto:welz@ese.eur.nl}{\texttt{welz@ese.eur.nl}}}
  
\date{%
    \textsc{Econometric Institute\\ Erasmus School of Economics}\\[2ex]%
    \today}
    
% definitions
\newcommand{\E}{\mathbb{E}}
\renewcommand{\P}{\mathbb{P}}
\newcommand{\R}{\mathbb{R}}
\newcommand{\N}{\mathbb{N}}
\newcommand{\V}{\mathbb{V}}
\newcommand{\D}{\mathcal{D}}
\newcommand{\e}{\text{e}}
\newcommand{\indic}{\mathds{1}}


\begin{document}

\maketitle

\section{Predictive Models}
\subsection{Risk Modeling}
I have implemented the risk modeling as in \cite{kent2020path} for both an ordinary and survival setup, where the latter is based on a completing risk Cox proportional subdistribution hazard  model \citep{fine1999proportional}, which generalizes the famous approach of \cite{cox1972} to multiple causes of failure. Below a summary.
\begin{itemize}
	\item Implementation of the first stage, both for the ordinary and survival setup, has been straightforward, as existing software for regularized regression with an elastic net penalty \citep{zou2005} could be used (\texttt{glmnet} and \texttt{fastcmprsk} packages). Note that no hierarchical penalty as in \cite{bien2013} is required here, since the first stage neither includes interaction effects nor a treatment assignment variable. 
	\item We cannot guarantee accurate inference on the coefficients of a risk model. While reliable estimates of the coefficients can be obtained, their confidence intervals will be too narrow due to the additional estimation uncertainty from the first stage. This problem is related to post-selection inference and inference in multi-stage models. Thus, obtaining reliable confidence intervals in risk models (in particular for survival models) is an interesting area of further (technical) research, and beyond the scope of CITRUS. Luckily, for the purposes of CITRUS, point estimates of the coefficients suffice.
	\item Risk models (especially in a survival setup!) are arguably not robust. We could use a robust regularized Median-of-Means-type estimator \citep{lecue2020} for the first stage and robust logistic regression in the second stage in the ordinary setup, but I'm not sure how well these will work in the survival setup due to a lack of literature. Again, further research is required here, which is beyond the scope of CITRUS. We could easily show that risk models are not robust to situations where the data are not identically distributed and outline this as a problem in CITRUS, but I think that offering a methodological solution is beyond the scope of this project (but still an interesting area of further research!).
	\item The issue with the discrepancy between the model description in \cite{rekkas2019} and \cite{kent2020path} has been solved.
	\item Overall, we are ready to run the risk models.
\end{itemize}


\subsection{Effect Modeling}
I am currently working on effect modeling as in \cite{kent2020path} with a hierarchical penalty as in \cite{bien2013}, both in an ordinary and survival context. Luckily, I found the useful reference \cite{du2021}, which adapts the method of \cite{bien2013} to the analysis of treatment effect heterogeneity, and I am currently implementing their proposed method. Below a summary.
\begin{itemize}
	\item We have the same situation for inference in effect modeling  as for risk modeling, namely that there does not yet exist a method for valid inference, hence this is an area of further research that is beyond the scope of CITRUS. Luckily, for the purposes of CITRUS, point estimates of the coefficients suffice.
	\item Effect models have the same robustness limitations as risk models.
	\item We are almost ready to run effect models, I should have an implementation by early next week.
\end{itemize}


\section{Considered models} 
\begin{itemize}
    \item one-variable-at-a-time analysis (can only estimate the $\textsf{ARTE}$);
    \item risk models and effect models, both with and without survival, as in \cite{kent2020path};
    \item causal random forest, both with and without survival (\citealp{athey2019grf} and \citealp{cui2021estimating});
    \item Double Machine Learning \citep{chernozhukov2018double} (can only estimate the $\textsf{ATE}$);
    \item Generic Machine Learning \citep{chernozhukov2020generic} (give qualitative evaluation of presence of treatment effect heterogeneity).
\end{itemize}


\section{Issues that we want to model}
\begin{itemize}
	\item Ability to capture nonlinear heterogeneity;
	\item Small samples;
	\item Improper randomization and unequal randomization (e.g.. 2-1 randomization);
	\item Robustness against false positives;
	\item Effects of missing data (demonstrate which imputation techniques are appropriate and which ones aren’t; e.g. \citealp{valdiviezo2015tree}) and extreme observations (heavy smokers; e.g. under or overreporting of smoking behavior);
	\item Low representation of risk factors. This means that we only have noisy information on important predictors. (which strictly speaking violates the unconfoundedness assumption).
	\item Variable transformation (some variables always get transformed in medicine)
	\item Different type of variables: rating-scale, continuous, categorical, and their encoding. And mix thereof.
	\item Categorization of continuous variables.
	\item Added value of survival models, vilation of proportional hazards assumption (e.g. \citealp{stensrud2020test}).
	\item For the simulation design, we might want to get inspiration from \cite{knaus2021machine}.
\end{itemize}


\section{Personal Notes}
It would be useful to have a self-written code of the optimizatition routines in \cite{friedman2010} and \cite{simon2011}, also for later projects in this PhD. I could implement them in \textsf{R}, but \textsf{R} will probably be too slow. Hence, we should consider C/C++ for this purpose. I however lack the experience in C/C++ in such applications, so it would probably take me a few days to get this to run.

If we want to take a deeper look into inference, \cite{guo2021} seems useful.

% stensrud2020test





\bibliography{bibliography}

\end{document}
