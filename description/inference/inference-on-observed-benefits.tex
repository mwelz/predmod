\documentclass{article}
\usepackage[utf8]{inputenc}
\usepackage{amsmath, amssymb, hyperref, mathtools}

\title{Statistical Inference on Observed Benefit Estimates}
\author{Max Welz}
\date{\today}

\begin{document}

\maketitle

\section{Setup}
Let there be $n$ i.i.d. binary outcome variables $Y_i, i=1,\dots,n$. If for sample $i$ we observe $Y_i=1$, we say that sample $i$ \textit{has the outcome}. If $Y_i=0$, then sample $i$ does not have the outcome. For the sake of brevity, we refer to $Y_i$ as all-cause mortality, meaning that $Y_i = 1$ denotes a death. Moreover, let the binary variable $W_i$ denote treatment assignment; that is, $W_i = 1$ if and only if sample $i$ received the treatment, and $W_i = 0$ otherwise. Suppose furthermore that the samples can be grouped into $m$ groups which we index by $1,2,\dots,m$. Denote the group membership of sample $i$ by $G_i \in \{1,\dots,m\} = \mathcal{G}$. Typically the grouping is done based on quantiles of the estimated baseline risk of having the outcome, but for the sake of generality, we treat the group membership as given here.

\section{Absolute Observed Benefit}
For each group $g \in \mathcal{G}$ calculate the group-level \textit{absolute} observed benefit $\widehat{aob}_g$ by
\begin{equation}
\begin{split}
    \widehat{aob}_g = &|\{ i : G_i = g, W_i = 1\}|^{-1} \sum_{\{ i : G_i = g, W_i = 1 \}} Y_i
    - \\
    &|\{ i : G_i = g, W_i = 0\}|^{-1} \sum_{\{ i : G_i = g, W_i = 0 \}} Y_i.
\end{split}
\label{eq:aob-def}
\end{equation}
Since the group-level absolute observed benefit is essentially the difference in the average group-level mortality, we can view it as an estimate of the group-level average treatment effect.

Due to this ``difference in means'' structure, we can use elementary statistical theory to construct confidence intervals for the absolute observed benefit $\widehat{aob}_g$. Concretely, we can construct a \textit{Welch t-test} for performing tests on $\widehat{aob}_g$. We use a Welch $t$-test rather than a classical $t$-test for the sake of generality: Unlike a classical $t$-test, Welch's test does not assume equal population variances of the two summands of $\widehat{aob}_g$ in equation \eqref{eq:aob-def}.

\section{Relative Observed Benefit}
For each group $g \in \mathcal{G}$ calculate the group-level \textit{relative} observed benefit $\widehat{rob}_g$ by the relative risk (or risk ratio)
\begin{equation}
    \widehat{rob}_g = \frac{|\{ i : G_i = g, W_i = 1\}|^{-1} \sum_{\{ i : G_i = g, W_i = 1 \}} Y_i
    }{|\{ i : G_i = g, W_i = 0\}|^{-1} \sum_{\{ i : G_i = g, W_i = 0 \}} Y_i}.
\label{eq:rob-def}
\end{equation}
Observe that the relative observed benefit corresponds to the ratio of the two summands that constitute the absolute observed benefit in equation \eqref{eq:aob-def} and can be viewed as a ratio of two proportions. The relative observed benefit estimates the theoretical group-level risk ratio
\begin{equation}
    rr_g = \frac{\mathbb{P}(Y_1 = 1 | W_1 = 1, G_1 = g)}{\mathbb{P}(Y_1 = 1 | W_1 = 0, G_1 = g)}.
\label{eq:rr_g}
\end{equation}

Constructing a confidence interval for the group-level relative observed benefit $\widehat{rob}_g$ is somewhat more involved than for the absolute observed benefit, because the ratio $\widehat{rob}_g$ usually does not asymptotically follow a normal distribution. Using the procedure suggested \href{https://sphweb.bumc.bu.edu/otlt/mph-modules/bs/bs704_confidence_intervals/bs704_confidence_intervals8.html}{\texttt{here}}, we construct confidence intervals for $\widehat{rob}_g$ by using a two-step procedure.

We first need some ancillary quantities. Within a given group $g \in \mathcal{G}$ let the constants
\begin{equation}
\begin{split}
    N_g^{(Y=1, W=1)} &= |\{i : G_i = g, Y_i = 1, W_i = 1\}|,
    \\
    N_g^{(Y=0, W=1)} &= |\{i : G_i = g, Y_i = 0, W_i = 1\}|,
    \\
    N_g^{(Y=1, W=0)} &= |\{i : G_i = g, Y_i = 1, W_i = 0\}|, \text{ and }
    \\
    N_g^{(Y=0, W=0)} &= |\{i : G_i = g, Y_i = 0, W_i = 0\}|
\end{split}
\end{equation}
count the number of treated deaths, treated survivors, untreated deaths, and untreated survivors, respectively.\footnote{Usually these counters are represented in a $2 \times 2$ contingency table as in \href{https://www.ncbi.nlm.nih.gov/pmc/articles/PMC2938757/}{\texttt{here}}.} Similarly, let
\begin{equation}
    N_g^{(W=1)} = |\{i : G_i = g, W_i = 1\}|
    \quad \text{and} \quad 
    N_g^{(W=0)} = |\{i : G_i = g, W_i = 0\}|
\end{equation}
count the number of treated and untreated samples, respectively, in group $g$. For a given group $g \in \mathcal{G}$, denote by 
\begin{equation}
    Z_g 
    =
    \frac{N_g^{(Y=0, W=1)}}{N_g^{(Y=1, W=1)}}
    \bigg/ N_g^{(W=1)}
    +
    \frac{N_g^{(Y=0, W=0)}}{N_g^{(Y=1, W=0)}}
    \bigg/ N_g^{(W=0)}
\end{equation}
the sum of the relative survival for the treatment and control group (normalized by the number of samples in each of these groups). 

Finally, for a given group $g \in \mathcal{G}$, a corresponding relative observed benefit $\widehat{rob}_g$, a significance level $\alpha \in (0,0.5)$, and a $(1-\alpha)$-quantile of the standard normal distribution $z_{1-\alpha / 2}$, denote the real-valued interval $\mathcal{C}^{\log(rr)}_g$ by 
\begin{equation}
    \mathcal{C}^{\log(rr)}_g
    =
    \Big[
    \log \big( \widehat{rob}_g\big) - z_{1-\alpha / 2} \sqrt{Z_g},\ 
    \log \big( \widehat{rob}_g\big) + z_{1-\alpha / 2} \sqrt{Z_g}
    \Big].
\label{eq:ci-log(rr)}
\end{equation}
The interval $\mathcal{C}'_g$ is a $(1-\alpha)$-confidence interval for $\log(rr_g)$. A simple exponential transformation of \eqref{eq:ci-log(rr)} yields the desired $(1-\alpha)$-confidence interval for $rr_g$:
\begin{equation}
\begin{split}
    \mathcal{C}_g^{rr}
    &=
    \Bigg[ 
    \exp\Big( \log \big( \widehat{rob}_g\big) - z_{1-\alpha / 2} \sqrt{Z_g} \Big),
    \exp\Big( \log \big( \widehat{rob}_g\big) + z_{1-\alpha / 2} \sqrt{Z_g} \Big)
    \Bigg]
    \\
    &=
   \Bigg[ 
   \widehat{rob}_g 
   \exp \Big(
   - z_{1-\alpha / 2} \sqrt{Z_g} \Big),\ 
   \widehat{rob}_g 
   \exp \Big(
    z_{1-\alpha / 2} \sqrt{Z_g} \Big)
    \Bigg].
\end{split}
\end{equation}

\section{Empirical Odds Ratio}
Define for the relevant group-level theoretical probabilities $p_g^{(1)} = \mathbb{P}(Y_1 = 1 | W_1 = 1, G_1 = g)$ and $p_g^{(0)} = \mathbb{P}(Y_1 = 1 | W_1 = 0, G_1 = g)$. Then, the relative risk in \eqref{eq:rr_g} can be written as $rr_g = p_g^{(1)} / p_g^{(0)}$. The theoretical group-level \textit{odds ratio} is defined as
\begin{equation}
    or_g = \frac{p_g^{(1)} / (1 - p_g^{(1)})}{p_g^{(0)} / (1 - p_g^{(0)})}.
\end{equation}

Define now the empirical probabilities by 
\begin{align*}
\hat{p}_g^{(1)} &= \frac{N_g^{(Y=1,W=1)}}{N_g^{(Y=1,W=1)} + N_g^{(Y=0,W=1)}} \quad \text{and}
\\
\hat{p}_g^{(0)} &= \frac{N_g^{(Y=1,W=0)}}{N_g^{(Y=1,W=0)} + N_g^{(Y=0,W=0)}}.
\end{align*}
Analogously to the theoretical group-level odds ratio $or_g$, the \textit{empirical} odds ratio $\widehat{or}_g$ is given by
\begin{equation}
    \widehat{or}_g = \frac{\hat{p}_g^{(1)} / (1 - \hat{p}_g^{(1)})}{\hat{p}_g^{(0)} / (1 - \hat{p}_g^{(0)})}.
\end{equation}

Similarly to the relative observed benefit, we cannot immediately construct a confidence interval for the theoretical group-level odds ratio $or_g$, but need a two-step procedure.
The variance of the estimator $\widehat{or}_g$ is given by
\begin{equation}
    V_g = \frac{1}{N_g^{(Y=1, W=1)}} + 
    \frac{1}{N_g^{(Y=0, W=1)}} +
    \frac{1}{N_g^{(Y=1, W=0)}} + 
    \frac{1}{N_g^{(Y=0, W=0)}}.
\end{equation}
For a given significance level $\alpha \in (0,0.5)$ and a $(1-\alpha)$-quantile of the standard normal distribution $z_{1-\alpha / 2}$, denote the real-valued interval $\mathcal{C}_g^{\log (or)}$ by
\begin{equation}
    \mathcal{C}^{\log(or)}_g
    =
    \Big[
    \log \big( \widehat{or}_g\big) - z_{1-\alpha / 2} \sqrt{V_g},\ 
    \log \big( \widehat{or}_g\big) + z_{1-\alpha / 2} \sqrt{V_g}
    \Big].
\end{equation}
The interval $\mathcal{C}_g^{\log (or)}$ is a $(1-\alpha)$-confidence interval for $\log(or_g)$. A simple exponential transformation yields he desired $(1-\alpha)$-confidence interval for $or_g$:
\begin{equation}
\begin{split}
    \mathcal{C}_g^{or}
    &=
    \Bigg[ 
    \exp\Big( \log \big( \widehat{or}_g\big) - z_{1-\alpha / 2} \sqrt{V_g} \Big),
    \exp\Big( \log \big( \widehat{or}_g\big) + z_{1-\alpha / 2} \sqrt{V_g} \Big)
    \Bigg]
    \\
    &=
   \Bigg[ 
   \widehat{or}_g 
   \exp \Big(
   - z_{1-\alpha / 2} \sqrt{V_g} \Big),\ 
   \widehat{or}_g 
   \exp \Big(
    z_{1-\alpha / 2} \sqrt{V_g} \Big)
    \Bigg].
\end{split}
\end{equation}

\end{document}
