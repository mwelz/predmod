\documentclass[11pt]{article}
\usepackage[utf8]{inputenc}
\usepackage{amsmath, bm, amssymb, hyperref, xcolor, caption, dsfont}

% link coloring
\hypersetup{
	hidelinks,
    colorlinks = true,
    linkbordercolor = {white},
    citecolor = blue,
    urlcolor = blue,
    linkcolor = blue,
    linktoc = all 
}

% layout
\usepackage[
  top=2cm,
  bottom=2cm,
  left=1.5in,
  right=1.5in,
  headheight=17pt, % as per the warning by fancyhdr
  includehead,includefoot,
  heightrounded, % to avoid spurious underfull messages
]{geometry} 

% Figure caption options
\captionsetup{labelfont = {bf, sc, color=blue}} % figure label 
\captionsetup{font = {singlespacing, small}} % caption

%%  bib settings
\usepackage{natbib}
\setcitestyle{authoryear} 
\bibliographystyle{apalike} 

% author information
\title{\textsc{Baseline Simulation}}
\author{Max Welz \\
  \href{mailto:welz@ese.eur.nl}{\texttt{welz@ese.eur.nl}}}
  
\date{%
    \textsc{Econometric Institute\\ Erasmus School of Economics}\\[2ex]%
    \today}
    
% definitions
\newcommand{\E}{\mathbb{E}}
\renewcommand{\P}{\mathbb{P}}
\newcommand{\R}{\mathbb{R}}
\newcommand{\N}{\mathbb{N}}
\newcommand{\V}{\mathbb{V}}
\newcommand{\D}{\mathcal{D}}
\newcommand{\e}{\text{e}}
\newcommand{\indic}{\mathds{1}}


\begin{document}
\maketitle
We here describe a baseline simulation design. This simulation design is intended to be as simple as possible. The rationale is that if a certain estimator does not perform well in this simple framework, it will most likely also not perform well in more complicated frameworks. Concretely, this simulation design entails:
\begin{itemize}
    \item No treatment effect heterogeneity;
    \item No noise;
    \item Randomized experiment;
    \item Perfect information;
    \item Linear data generating process;
    \item Identically and independently distributed Gaussian covariates;
    \item All deceasing individuals die at the exact same time.
\end{itemize}

\paragraph{Sampling of covariates.} Let $n$ be the number of observations and $p$ be the number of covariates. The $p$-dimensional covariate vector of observation $i, X_i,$ is standard normally distributed. The binary treatment assignment variable $W_i$ assigns the treatment randomly. Mathematically, for all $i = 1,\dots,n$,
\[
    X_i \sim \mathcal{N}_p(0, I)
    \quad  \text{ and }\quad 
    W_i \sim \textsf{Bernoulli}(0.5).
\]

\paragraph{Predictive function.}
Suppose that $\beta_0\in\R$ is an intercept term and that $\beta\in\R^p$ is a vector of coefficients associated with the covariates. Let $\theta\in\R$ be the \textit{treatment effect parameter}, which governs the strength of the treatment effect. Since $\theta$ is a constant, there is no treatment effect heterogeneity. An effective treatment strategy $W_i$ is characterized by $\theta \neq 0$. We assume that the \textit{predictive function}, $\eta_i(W_i)$, a function of $W_i$, obeys
\[
    \eta_i(W_i) := \eta(W_i, X_i)
    =
    \beta_0 + X_i^\top \beta + \theta W_i.
\]

Hence, the predictive function is assumed to be linear in its parameters $(\beta_0, \beta, \theta)$.

\paragraph{Mortality risk.} We emulate a medical trial. We are interested in the effect of treatment $W_i$ on the (binary) mortality $Y_i$, where $Y_i=1$ if individual $i$ dies during the trial, and $Y_i=0$ if it survives. The probability that individual $i$ with treatment assignment status $W_i$ dies is called the \textit{mortality risk}, which is denoted by $P_i(W_i)$ and defined as
\[
P_i(W_i) := P(W_i, X_i) = \P [Y_i=1 | W_i, X_i]
= F_{logistic}\Big( \eta_i(W_i) \Big),
\]
where $F_{logistic}$ is the distribution function of the logistic distribution.

\paragraph{Potential Outcomes.} 
The potential outcomes $Y_i(1)$ and $Y_i(0)$ denote the outcome if individual $i$ is treated and if it is \textit{not} treated, respectively. Since the outcome is binary, we assume that
\[
    Y_i(W_i) \sim \textsf{Bernoulli} \Big( P_i(W_i) \Big). 
\]
The \textit{observed outcome} $Y_i$ is defined by
\[
    Y_i = 
    \begin{cases}
    Y_i(1) \quad \text{ if } W_i =1, \\
    Y_i(0) \quad \text{ if } W_i =0.
    \end{cases}
\]

\paragraph{Time at risk.} We assume that the trial takes 10 years and that all deceasing individuals die after 7 years. That is, individual $i$'s \textit{time-at-risk}, $T_i$, is given by
\[
    T_i = 7 + 3\indic_{\{ Y_i = 0\}}.
\]

\paragraph{Quantities of interest.}
We are interested in estimating the following quantities by using the observations $\{(X_i, Y_i, W_i, T_i)\}_{i=1}^n$:
\begin{itemize}
    \item Treatment effect parameter $\theta$;
    \item 95\% coverage of $\theta$;
    \item Average treatment effect: $\textsf{ATE} = \frac{1}{n} \sum_{i=1}^n \big(Y_i(1) - Y_i(0)\big)$;
    \item Average relative treatment effect: $\textsf{ARTE} = \frac{1}{n} \sum_{i=1}^n \big( Y_i(1) \big/ Y_i(0) \big)$;
    \item Calibration curve, both relative and absolute.
\end{itemize}

\paragraph{Parameter choices}. We fix the parameters as follows.
\begin{itemize}
    \item $n=10,000$ and $p=10$;
    \item $\theta = -1$ (that is, the treatment is effective);
    \item $\beta_0 = 0$, and, for $j=1,\dots,p,$
    \[
        \beta_j = \begin{cases}
        1 &\text{ if $j$ is even}\\
        -1  &\text{ if $j$ is odd}.
        \end{cases}
    \]
\end{itemize}

\end{document}
